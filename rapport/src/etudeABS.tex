\section{Study of price as a function of the prescene of ABS-technology} % (fold)
\label{sec:price_abs}

As seen in Section \ref{sec:Char_Corr} cars with ABS seem to generaly have a higher price that cars without this technology. We will further analyze how well \textit{price} can be explained by the \textit{ABS}.

\noindent
To better understand how the \textit{R} software works we created two simple regression models with all variables provided with \textit{ABS} as qualative and quantitative variable. In implementation qualitative means that the variable is given a numerical value, 1 representing the presence of ABS or 0 otherwise. The quantative variable is a logical data type meaning it can be either \textit{TRUE} or \textit{FALSE}. No difference can be seen in the \textit{summary} function (which provides values and different significance metrics for each coefficient) between those two models. It may be pertinant remember that the logical variable cannot be calculated upon, e.g. we cannot calculate the it's mean value. This should not make a difference in the final model as regression requires that the logical variable be implemented as a dummy variable (i.e. qualitavitely) anyways.

\noindent
The simplest of regressions can be a good way to analyze the explanatory power of \textit{ABS}. We created a regression model of \textit{price} as only a function of \textit{ABS} corresponding to the following model.

\begin{equation}
	price = \beta_0 + \beta_1 \times ABS + \epsilon
\end{equation}

\noindent
Here $\beta_0$ is the intercept and $\beta_1$ is the slope of the equation. $\epsilon$ is the residual variable. Regression on our dataset resulted in the following estimations. $\hat{\beta}_0$ = 3.45363 and $\hat{\beta}_1$ = -0.08321. It is clear that the variable \textit{ABS} does not account for any of the variance of the price, since it's coefficent of deterimination, $R^2$ is very low (0.00097). That means that the model does not do a very good job predicting the price. This fact is also reflected by the p-value for \textit{ABS}, $p=0.686$. the p-value is the probility of obtaining a result as more extreme given the null hypothisis (which in this case is that the coefficient $\beta_1 = 0$.) For these reasons it is clear that \textit{ABS} on it's own explains the price very poorly. Arguably, the intercept alone would be a better model as the model
would just create a nearly constant line.

% section price_abs (end)