\section{Study of Price as a Function of All Variables} % (fold)
\label{sec:study_of_price_as_a_function_of_all_variables}

In this section, we will first calculate a linear regression model 
that takes all variables of the dataset into account in order to
predict the price of a used car. In the following, we will use different 
methods do decide on which variables will be used in our ``final model''. 
Therefore, different factors will be taken into consideration, such as:

\begin{itemize}
	\item the \emph{``R-squared''} value showing how well the predictions fit the actual data and detected \emph{high biais},
	\item factors making sure the data is not \emph{overfitting}, such as 
	\emph{BIC}, \emph{AIC} and \emph{adjusted ``R-squared''}
	\item and factors using cross-validation methods in order to check 
	whether the model is useful for data the model was not overfitted on our dataset 
	(\emph{BIC}, \emph{AIC}).
\end{itemize}

\subsection{Prelimenary Selection of Different Models} % (fold)
\label{sub:prelimenary_selection_of_different_models}

/noindent
The first model we want to test later is simply the model that uses all 
variables in order to predict the price: 

\begin{equation}\label{eq:model1}
	M_1 = lm(price \sim\ .)
\end{equation}

/noindent
The second model is obtained by looking at the p-values of every input 
variable using:

\begin{lstlisting}[caption={summary() in R},label={lst:summary_func}]
	summary(Model1)
\end{lstlisting}

/noindnet
There are three variables that have a significant lower 
p-value than the other variables, \emph{age, km and ageop2}.
Therefore our second model is as follows:

\begin{equation}\label{eq:model2}
	M_2 = lm(price \sim\ age + km + ageop2)
\end{equation}

/noindnet
Our third model was achieved by using stepFunction: ``stepAIC''. 
This function calculates the \emph{AIC} value of the model at every step and 
removes all the variables, so that the remaining model has a lower AIC and is
therefore a ``better fit''.
Using this method, we get the following model:

\begin{equation}\label{eq:model3}
	M_3 = lm(price \sim\ age + km + ageop2 + kop2)
\end{equation}

/noindnet
Our fourth and last model was created using a model of preselected 
variables and executing the ''stepAIC'' function on this model in order to choose additional variables. The 
It is in some way a mixture of the methods used to obtain model2 and model3. 
We preselect the variables \emph{age, km} since they show by far the highest p-values (see listing \ref{lst:summary_func}). Applying the ``stepAIC'' function, we get: 

\begin{equation}\label{eq:model4}
	M_4 = lm(price \sim\ age + km + ageop2 + kop2)
\end{equation} 

/noindent
This is the same model as model 3 and can therefore be discarded.

\subsection{Validation of the Models} % (fold)
\label{sub:validation_of_the_models}

/noindnet
First, we want to calculate the different \emph{``R-squared''} values to select models having low biases when predicting the data. 
For all three models we get similiar results.

\begin{table}[ht]
\begin{center}
\begin{tabular}{ |c|c|c|c| } 
 \hline
 & $M_1$ & $M_2$ & $M_3$ \\ 
 \hline
 $R^2$ &  0.65833 & 0.64338 & 0.65581 \\ 
 \hline
\end{tabular}
\end{center}
\end{table}
% subsection validation_of_the_models (end)

/noindent
The closer \emph{``R-squared''} is to 1, the better it is. Obviously, the 
first model is the closest to one since it uses all the variables for its model. We can see though that all \emph{``R-squared''} are very close making it difficult to pick a clear winner. Further analysis is needed.

/noindnet
To get in-depth analysis, we will use the \textit{CV} function of the package \textit{forecast}. This function calculates the \emph{BIC}, \emph{AIC}, \emph{AICc} and \emph{adjusted ``R-squared''} for every model and returns 
a so-called \emph{PRESS} value that takes into consideration all the factors mention above in the first part of Section \ref{sec:study_of_price_as_a_function_of_all_variables} (especially the second and third factor). The lowest \emph{PRESS} suggest that the corresponding model is best relative to the others. The result of the \textit{CV} function is displayed in the table below:

\begin{figure}[H]
  \begin{center}
    \includegraphics[scale=0.7]{./img/CV_analysis.png}
    \end{center}
  \caption{\textit{PRESS} anaysis of the models.}
  \label{fig:PRESS}
\end{figure}

As we can see, we get the lowest CV for the third model. Our model from the preliminary selection in section \ref{sub:prelimenary_selection_of_different_models}.

% subsection prelimenary_selection_of_different_models (end)

% section study_of_price_as_a_function_of_all_variables (end)

