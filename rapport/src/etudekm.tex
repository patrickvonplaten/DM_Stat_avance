\section{Study of Price as a Function of Mileage} % (fold)
\label{sec:price_km}

\subsection{Simple Regression} % (fold)
\label{sub:simple_regression}

A simple regression of \textit{price} on milage represented by the variable \textit{km} shows quite different results. As in Section \ref{sec:price_abs}, we begin by performing a simple regression using the equation

\begin{equation}\label{eq:price_km}
	price = \beta_0 + \beta_1 \times km + \epsilon
\end{equation}

\noindent
where $\beta_0$ is the intercept, $\beta_1$ is the slope and $\epsilon$ is the residual term. Applying this to our data results in $\beta_0$ and $\beta_1$ values of 5.559202 and -0.016051. Using the same analysis and reasoning as in \ref{sec:price_abs}, $km$ is obviously a better parameter to describe the price of a car. The coefficent of detirmination $R^2$ is much higher (0.3317) and the p-value is much lower ($<2e^{16}$), suggesting that \textit{km} should be in model explaining the price.

% subsection simple_regression (end)

\subsection{Inference from Simple Model} % (fold)
\label{sub:inference_from_simple_model}

We will now using the simple model from the previous section create confidence intervals of prices for cars with difference milages. This implemented in \textit{R} using the \textit{predict} function. For a cars with a milages of 50 000 km and 135 000 km a confindence intervals at the defult significance level 95\% are shown in Table \ref{tab:confint_km}.

\begin{table}[H]
\centering
\begin{tabular}{||c |c c c||} 
 \hline
 Mileage & fit & lower & upper \\ [0.5ex]
 \hline
 50 000 km & 4.756639 & 4.426392 & 5.086886 \\ 
 136 000 km & 3.392284 & 3.238505 & 3.546063 \\
 \hline
\end{tabular}
\caption{Confindence intervals at 95\% confidence level for cars with milages of 50 000 km and 135 000 km}
\label{tab:confint_km}
\end{table}

\noindent
We see that the confidence intervall for a car that has 135 000 km milage is much smaller than the other at the same sig. level. This implies that the variance of the price for cars having 135 000 km is lower that can be justified by the fact that the price of cars already having a high usage (in terms of km) are naturally lower and don't vary as much. Whereas for cars having lower usage (50 000 km) the price can vary much more since other factors play a more important role in determining its price (car type, car performance.) Also, it should be noted that 135 000 km is the mean of all of the data. Since the sample follows a t-distribution, it is normal that there is more data around 135 000 km (the distribution is denser). Therefore the 95\% confidence intervall has a shorter lenght than the one at 50 000 km.
% subsection inference_from_simple_model (end)

\subsection{Determination of Mystery Variable \textit{kop1}} % (fold)
\label{sub:det_myst_var}

The data set given includes the variable \textit{kop1} given wit hout explaination of what it charectarizes. In fact, it reduces to the variable \textit{km} as it is a centered and reduced version of it.

\begin{equation}\label{eq:1}
	kop1 \approx \frac{km - mean(km)}{SD(km)}
\end{equation}

\noindent
In fact, simple regression models based on \textit{kop1} and \textit{km} result in the exact same model. It can be shown that the models are logically the same since $R^2$ and other important functions are equal. Algerbrically this can be show through the following reasoning. The regression model is given by the following equation:

\begin{equation}\label{eq:price_kop1}
	price = \gamma_0 + \gamma_1 \times kop1 + e
\end{equation}

\noindent
Where $\gamma_i$ are the coefficents and $e$ is the residual. Using relationship \ref{eq:1} the regression model reduces to

\begin{equation}
	\begin{split}
	 price & \approx \gamma_0 + \gamma_1 \times \frac{km - mean(km)}{SD(km)}+ e \\ &
	 = (\gamma_0 +  \frac{\gamma_1 \times mean(km)}{SD(km)}) + \frac{\gamma_1}{SD(km)} \times km + e.
	 \end{split}
\end{equation}

\noindent
Given that $mean(km)$ and $SD(km)$ are constants, we note that this is this approximately equivant with the regression model \ref{eq:price_km} as $\beta_0 \approx \gamma_0 + \frac{\gamma_1 \times mean(km)}{SD(km)}$, $\beta_1 \approx \frac{\gamma_1}{SD(km)}$ and the residuals $\epsilon \approx e$. It is obvious that centering and reducing shouldn't change anything. The regression looks at how changes in the explainatory variables effect the response variable (here \textit{price}). As simple regression models with \textit{km} and \textit{kop1} are essentially regressing on the same random variable they should result in equally good models with the same metrics such as $R^2$ which are equivelant (0.3317) and the p-value for the slope ($<2e^{16}$).

% subsection det_myst_var (end)

\subsection{Polynomial Regression Model} % (fold)
\label{sub:poly_km}

We will now analyze the performance of a polynomial regression model. Given the third degree equation \ref{eq:price_km_poly} and our data, the results are. 

\begin{equation}\label{eq:price_km_poly}
	price = \beta_0 + \beta_1 \times km + \beta_2 \times km^2 + \beta_1 \times km^3 
	+ \epsilon
\end{equation}

\noindent
This is a linear model as it is defined given a random sample $$(Y,X_1,\ldots,X_N)$$ the relation between the observations $Y$ and the independant variables $X_1,\ldots,X_N$ is formulated as $Y = b_0 + b_1*f_1(X_1) + \ldots + b_N*f_N(X_N)$, where $F_i$ may be non-linear funcions.

\noindent
It is interesting to note that the p-values for $\beta_1$, $\beta_2$ and $\beta_3$ are quite high 0.132, 0.839 and 0.766 respectively. The $R^2 = 0.366$ value is slightly better than the simple model. Despite this, it can be concluded that the the third degree polynomial explains the price of cars relatively poorly.
% subsection poly_km (end)

\subsection{ANOVA} % (fold)
\label{sub:anova}

ANOVA or Analysis of Variance is now performed on the polynomial model. It compares the means of several grouop in order to among other things conclude the significance of each variable. We have implemented this on two regression models. The polynomial model form the previous section as well as a model in which the order of the polynomial terms is reversed. We will refer to them here as Model 1 and Model 2 respectively. The results are shown in the Table \ref{tab:3}.

\begin{equation}
	price = \alpha_0 + \alpha_1 \times km^3 + \alpha_2 \times km^2 + \alpha_1 \times km +e
\end{equation}

\begin{table}[H]
\centering
\begin{tabular}{||c |c c c||}
 \hline
 Variable & $km$ & $km^2$ & $km^3$ \\ [0.5ex]
 \hline
 Model 1 & $< 2 e^{-16}$ & 0.00313 & 0.76573 \\
 Model 2 & 0.132 & $7.444 e^{-9}$ & $2.104 e^{-12}$ \\
 \hline
\end{tabular}
\caption{P-values for each variable in Model 1 and Model 3}
\label{tab:3}
\end{table}

\noindent
When applying the anova method to the two models, we can see different results even though the three input variables are the same. This is due to the order the input varibles are taken in as an input. In our case, \textit{anova()} determines how much variance is explained by the first entry (\textit{km} e.g.) and tests its significance, then what portion of the remaining variance is explained by the next variable ($km^2$) and tests its significance and so forth. Thus, the remaining portion will differ depending on the first variable being inserted and therefore  different significances (p-values) is the result.

% subsection anova (end)

\subsection{Variance Inflation Factors} % (fold)
\label{sub:variance_inflation_factors}



In order to calculate the vif of a model, we have to take the three input variables 
being in our case $km$, $km^2$ and $km^3$ and do a linear regression for all each
using the other remaining two variables as the input variables (e.g. for $km
$: $km ~ I(km^2) + I(km^3)$). Then we take the coefficient of determination of each model ($R^2$) and apply the formula $1 / ( 1 - R^2 )$. This results in the variance inflation factor (VIF) of each variable. For our dataset the \textit{vif} function from the \textit{car} package resulted in the values shown in Table \ref{tab:vif}.

\begin{table}[H]
\centering
\begin{tabular}{||c ||c c c||}
 \hline
 Variable & $km$ & $km^2$ & $km^3$ \\ [0.5ex]
 \hline
 VIF & $ 165.3163 $ & 787.3501 & 262.5217 \\
 \hline
\end{tabular}
\caption{VIF values for each variable in the polynomial model.}
\label{tab:vif}
\end{table}

\noindent
It can clearly be seen that all the VIF values are high meaning that the $R^2$ is relatively close to one. That implies that the function predicts the actual values quite well. The largest VIF is that of $km^2$, which is logical because $km^2$ can easily be modeled by summing up  $(factor * kw)$ and $(km^3 / factor)$ where as it is harder to model \texit{km} by taking its square and cube because their functions are both bigger than \textit{km}.

% subsection variance_inflation_factors (end)

